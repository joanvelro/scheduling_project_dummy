%% Generated by Sphinx.
\def\sphinxdocclass{report}
\documentclass[letterpaper,10pt,english]{sphinxmanual}
\ifdefined\pdfpxdimen
   \let\sphinxpxdimen\pdfpxdimen\else\newdimen\sphinxpxdimen
\fi \sphinxpxdimen=.75bp\relax
\ifdefined\pdfimageresolution
    \pdfimageresolution= \numexpr \dimexpr1in\relax/\sphinxpxdimen\relax
\fi
%% let collapsible pdf bookmarks panel have high depth per default
\PassOptionsToPackage{bookmarksdepth=5}{hyperref}

\PassOptionsToPackage{warn}{textcomp}
\usepackage[utf8]{inputenc}
\ifdefined\DeclareUnicodeCharacter
% support both utf8 and utf8x syntaxes
  \ifdefined\DeclareUnicodeCharacterAsOptional
    \def\sphinxDUC#1{\DeclareUnicodeCharacter{"#1}}
  \else
    \let\sphinxDUC\DeclareUnicodeCharacter
  \fi
  \sphinxDUC{00A0}{\nobreakspace}
  \sphinxDUC{2500}{\sphinxunichar{2500}}
  \sphinxDUC{2502}{\sphinxunichar{2502}}
  \sphinxDUC{2514}{\sphinxunichar{2514}}
  \sphinxDUC{251C}{\sphinxunichar{251C}}
  \sphinxDUC{2572}{\textbackslash}
\fi
\usepackage{cmap}
\usepackage[T1]{fontenc}
\usepackage{amsmath,amssymb,amstext}
\usepackage{babel}



\usepackage{tgtermes}
\usepackage{tgheros}
\renewcommand{\ttdefault}{txtt}



\usepackage[Bjarne]{fncychap}
\usepackage{sphinx}

\fvset{fontsize=auto}
\usepackage{geometry}


% Include hyperref last.
\usepackage{hyperref}
% Fix anchor placement for figures with captions.
\usepackage{hypcap}% it must be loaded after hyperref.
% Set up styles of URL: it should be placed after hyperref.
\urlstyle{same}

\addto\captionsenglish{\renewcommand{\contentsname}{Scheduler Contents:}}

\usepackage{sphinxmessages}
\setcounter{tocdepth}{3}
\setcounter{secnumdepth}{3}


\title{GEA}
\date{Mar 15, 2022}
\release{v.0.1}
\author{Capgemini Engineering \sphinxhyphen{} Hybrid Intelligence}
\newcommand{\sphinxlogo}{\vbox{}}
\renewcommand{\releasename}{Release}
\makeindex
\begin{document}

\pagestyle{empty}
\sphinxmaketitle
\pagestyle{plain}
\sphinxtableofcontents
\pagestyle{normal}
\phantomsection\label{\detokenize{index::doc}}


\noindent\sphinxincludegraphics[width=500\sphinxpxdimen]{{logos}.png}

\sphinxAtStartPar
The scheduler optimization module is formulated as a Mixed\sphinxhyphen{}Integer Programming problem (MIP) using the framework ORTools and Python as programming
language.
\begin{equation*}
\begin{split}min_{x \in \mathbb{R}, y \in (0,1)} \ f(x,y)\end{split}
\end{equation*}\begin{equation*}
\begin{split}st: \ g(x,y) <=0\end{split}
\end{equation*}
\sphinxAtStartPar
The module evaluates whether a certain plant configuration is able to satisfy a certain product demand within a given time frame (both also given as an input parameter)
according to some solution criterion to be clarified below.

\sphinxAtStartPar
Within the  scheduler module, an initial pre\sphinxhyphen{}check (input data analyzer) is executed before starting the scheduling problem to identify
possible data inconsistencies and will communicate the possible warnings or errors founded in the output of the scheduler.

\sphinxAtStartPar
If there exists a feasible schedule for the input plant configuration, the scheduler provides a schedule\sphinxhyphen{}type output, which will later be used
to construct one Gantt chart (the construction is out of this scope).

\sphinxAtStartPar
If no feasible schedule exists, then the relaxed schedule\sphinxhyphen{}type output is provided, and the list of the equipment involved with their utilization
ratios will be provided. This will be obtained  by relaxing certain constraints until a feasible problem can be reached to follow the same procedure
as with bottleneck identification.

\sphinxAtStartPar
If the input data instance is not consistent (exist errors) then no schedule\sphinxhyphen{}type output is provided. The error message will indicate clearly the
reason of failure.

\sphinxAtStartPar
The output will contain a list of possible warnings and error (data inconsistencies) messages.

\sphinxAtStartPar
This version includes:
\begin{itemize}
\item {} 
\sphinxAtStartPar
Batch/continuous processes.

\item {} 
\sphinxAtStartPar
Already running equipment.

\item {} 
\sphinxAtStartPar
Multi\sphinxhyphen{}product/Multi\sphinxhyphen{}workflows.

\end{itemize}

\sphinxAtStartPar
This version DOES NOT includes:
\begin{itemize}
\item {} 
\sphinxAtStartPar
CIPs.

\item {} 
\sphinxAtStartPar
Product Order.

\item {} 
\sphinxAtStartPar
Clusters.

\item {} 
\sphinxAtStartPar
Split Workflows.

\end{itemize}


\chapter{Scheduler Module Documentation}
\label{\detokenize{index:scheduler-module-documentation}}

\section{Model Data}
\label{\detokenize{index:module-src.model_data}}\label{\detokenize{index:model-data}}\index{module@\spxentry{module}!src.model\_data@\spxentry{src.model\_data}}\index{src.model\_data@\spxentry{src.model\_data}!module@\spxentry{module}}\phantomsection\label{\detokenize{index:module-0}}\index{module@\spxentry{module}!src.model\_data@\spxentry{src.model\_data}}\index{src.model\_data@\spxentry{src.model\_data}!module@\spxentry{module}}\index{ModelData (class in src.model\_data)@\spxentry{ModelData}\spxextra{class in src.model\_data}}

\begin{fulllineitems}
\phantomsection\label{\detokenize{index:src.model_data.ModelData}}\pysigline{\sphinxbfcode{\sphinxupquote{class\DUrole{w}{  }}}\sphinxcode{\sphinxupquote{src.model\_data.}}\sphinxbfcode{\sphinxupquote{ModelData}}}
\sphinxAtStartPar
\sphinxstyleemphasis{Model Data}

\sphinxAtStartPar
This class defines the data model of the input instance. It contains the main attributes that the scheduler
requires.It also contain the different methods required to work with the attributes.
\begin{quote}
\begin{description}
\item[{Attributes:}] \leavevmode
\sphinxAtStartPar
schedule\_configs:     Dictionary that contain the schedule configuration:
\begin{quote}

\sphinxAtStartPar
objective:          objective to consider (1:makespan, 2:just\sphinxhyphen{}in\sphinxhyphen{}time). String.

\sphinxAtStartPar
product\_order:      order of the products to be scheduled. List:

\sphinxAtStartPar
starting\_date:      Starting date to consider in the schedule. Timestamp (dd\sphinxhyphen{}mm\sphinxhyphen{}yyyy hh:mm:ss).

\sphinxAtStartPar
max\_time\_horizon:   Max. number of days to consider in the schedule. Integer.

\sphinxAtStartPar
time\_resolution:    Time resolution of the schedule (1\sphinxhyphen{}30 min). Integer

\sphinxAtStartPar
plant\_ID:    Identifier for the plant configuration. It wil be used for the schedule\_ID. String
\end{quote}

\sphinxAtStartPar
equipment: Dictionary that contains the information of the equipment involved in the schedule.
Each register is a equipment.
\begin{quote}

\sphinxAtStartPar
equipment\_ID: Identifier for the equipment.

\sphinxAtStartPar
no\_inputs:  Number of inputs.

\sphinxAtStartPar
no\_outputs: Number of outputs.

\sphinxAtStartPar
batch\_max: Maximum batch size.

\sphinxAtStartPar
batch\_min: Minimum batch size.

\sphinxAtStartPar
calendar: Equipment\_calendar. Working horus. format: {[}24,7{]}.
\end{quote}

\sphinxAtStartPar
demand: Dictionary that contains the information of the product demand:
\begin{quote}

\sphinxAtStartPar
product\_ID: Identifier for th product. Is a list.

\sphinxAtStartPar
due\_data:   Dictionary that contain the due\_date for each product.

\sphinxAtStartPar
amount: Dictionaty that contain the quantity demanded for each product
\end{quote}

\sphinxAtStartPar
workflows: Dictionary that contains the information related to recipe/workflows:
\begin{quote}

\sphinxAtStartPar
workflow\_ID:    Identifier for the workflow.

\sphinxAtStartPar
input\_product:  Input product/s. {[}String{]}.

\sphinxAtStartPar
output\_product: Ouput product/s. {[}String{]}.

\sphinxAtStartPar
recipe: Dictionary that contains the recipes (workflow graph)
\begin{quote}

\sphinxAtStartPar
node: Dictionary that contains:
\begin{quote}

\sphinxAtStartPar
node\_ID

\sphinxAtStartPar
subprocess

\sphinxAtStartPar
duration\_type

\sphinxAtStartPar
duration
\end{quote}

\sphinxAtStartPar
edge:   Dictionary that contains:
\begin{quote}

\sphinxAtStartPar
node\_origin\_ID

\sphinxAtStartPar
node\_destination\_ID

\sphinxAtStartPar
subprocess\_destination

\sphinxAtStartPar
subprocess\_origin

\sphinxAtStartPar
product\_origin

\sphinxAtStartPar
product\_destination

\sphinxAtStartPar
flow\_rate

\sphinxAtStartPar
delay

\sphinxAtStartPar
connection: Type of connection (SS, FF, FS, SF)
\end{quote}
\end{quote}
\end{quote}

\end{description}
\end{quote}

\end{fulllineitems}



\section{Scheduler Response}
\label{\detokenize{index:module-src.scheduler_response}}\label{\detokenize{index:scheduler-response}}\index{module@\spxentry{module}!src.scheduler\_response@\spxentry{src.scheduler\_response}}\index{src.scheduler\_response@\spxentry{src.scheduler\_response}!module@\spxentry{module}}\phantomsection\label{\detokenize{index:module-1}}\index{module@\spxentry{module}!src.scheduler\_response@\spxentry{src.scheduler\_response}}\index{src.scheduler\_response@\spxentry{src.scheduler\_response}!module@\spxentry{module}}\index{SchedulerResponse (class in src.scheduler\_response)@\spxentry{SchedulerResponse}\spxextra{class in src.scheduler\_response}}

\begin{fulllineitems}
\phantomsection\label{\detokenize{index:src.scheduler_response.SchedulerResponse}}\pysigline{\sphinxbfcode{\sphinxupquote{class\DUrole{w}{  }}}\sphinxcode{\sphinxupquote{src.scheduler\_response.}}\sphinxbfcode{\sphinxupquote{SchedulerResponse}}}
\sphinxAtStartPar
\sphinxstyleemphasis{Scheduler Response}

\sphinxAtStartPar
This class defines the schedule data class of the scheduler engine. It contains the necessary information to
build the schedule gantt chart.
\begin{quote}
\begin{description}
\item[{Attributes:}] \leavevmode
\sphinxAtStartPar
schedule\_status:         Indicate the status of the schedule (feasible, not\_feasible or not\_consistent)

\sphinxAtStartPar
equipment\_unfeasible:    It is a dictionary with all the equipment involved in the schedule with the
utilization ratios (UR).

\sphinxAtStartPar
status\_data\_analyzer:    It is a dictionary with warning and errors messages

\sphinxAtStartPar
schedule\_ID:             Identifier for the schedule (string)

\sphinxAtStartPar
schedule\_gantt:          Dictionary that contains:
\begin{quote}

\sphinxAtStartPar
equipment\_ID:        Equipment involved

\sphinxAtStartPar
workflow\_ID:         Workflow associated

\sphinxAtStartPar
product\_order:       Product Order associated

\sphinxAtStartPar
work\_order:          Work Order associated

\sphinxAtStartPar
job\_order:           job\_order associated

\sphinxAtStartPar
product:             Product associated

\sphinxAtStartPar
subprocess:          Subprocess

\sphinxAtStartPar
start:               Start timestep

\sphinxAtStartPar
end:                 End timestep

\sphinxAtStartPar
volume:              Quantity of product
\end{quote}

\end{description}
\end{quote}

\end{fulllineitems}



\section{Scheduler Engine}
\label{\detokenize{index:module-src.engine}}\label{\detokenize{index:scheduler-engine}}\index{module@\spxentry{module}!src.engine@\spxentry{src.engine}}\index{src.engine@\spxentry{src.engine}!module@\spxentry{module}}\phantomsection\label{\detokenize{index:module-2}}\index{module@\spxentry{module}!src.engine@\spxentry{src.engine}}\index{src.engine@\spxentry{src.engine}!module@\spxentry{module}}\index{SchedulerEngine (class in src.engine)@\spxentry{SchedulerEngine}\spxextra{class in src.engine}}

\begin{fulllineitems}
\phantomsection\label{\detokenize{index:src.engine.SchedulerEngine}}\pysiglinewithargsret{\sphinxbfcode{\sphinxupquote{class\DUrole{w}{  }}}\sphinxcode{\sphinxupquote{src.engine.}}\sphinxbfcode{\sphinxupquote{SchedulerEngine}}}{\emph{\DUrole{n}{data}\DUrole{p}{:}\DUrole{w}{  }\DUrole{n}{{\hyperref[\detokenize{index:src.model_data.ModelData}]{\sphinxcrossref{src.model\_data.ModelData}}}}}}{}
\sphinxAtStartPar
\sphinxstyleemphasis{Scheduler Engine}

\sphinxAtStartPar
This class defines the scheduler optimization engine.
\begin{quote}
\begin{description}
\item[{Attributes:}] \leavevmode
\sphinxAtStartPar
results:     Results of the pyomo execution.

\sphinxAtStartPar
name:        Name of the model.

\sphinxAtStartPar
data:        Model data instance.

\sphinxAtStartPar
response:    Scheduler instance.

\end{description}
\end{quote}
\index{execute() (src.engine.SchedulerEngine method)@\spxentry{execute()}\spxextra{src.engine.SchedulerEngine method}}

\begin{fulllineitems}
\phantomsection\label{\detokenize{index:src.engine.SchedulerEngine.execute}}\pysiglinewithargsret{\sphinxbfcode{\sphinxupquote{execute}}}{}{}
\sphinxAtStartPar
\sphinxstyleemphasis{Execute}

\sphinxAtStartPar
This method execute the engine module which comprise the following sub\sphinxhyphen{}methods:
\begin{itemize}
\item {} 
\sphinxAtStartPar
Data Analyzer: Check the integrity of the input data instance.

\item {} 
\sphinxAtStartPar
Build Model: Build the engine scheduler model.

\item {} 
\sphinxAtStartPar
Solve: Solve the optimziation model.

\item {} 
\sphinxAtStartPar
Build Solution: Invoke the scheduler data factory to build the schedule output solution

\end{itemize}

\end{fulllineitems}


\end{fulllineitems}



\section{Model Data Factory}
\label{\detokenize{index:module-src.model_data_factory}}\label{\detokenize{index:model-data-factory}}\index{module@\spxentry{module}!src.model\_data\_factory@\spxentry{src.model\_data\_factory}}\index{src.model\_data\_factory@\spxentry{src.model\_data\_factory}!module@\spxentry{module}}\phantomsection\label{\detokenize{index:module-3}}\index{module@\spxentry{module}!src.model\_data\_factory@\spxentry{src.model\_data\_factory}}\index{src.model\_data\_factory@\spxentry{src.model\_data\_factory}!module@\spxentry{module}}\index{ModelDataFactory (class in src.model\_data\_factory)@\spxentry{ModelDataFactory}\spxextra{class in src.model\_data\_factory}}

\begin{fulllineitems}
\phantomsection\label{\detokenize{index:src.model_data_factory.ModelDataFactory}}\pysiglinewithargsret{\sphinxbfcode{\sphinxupquote{class\DUrole{w}{  }}}\sphinxcode{\sphinxupquote{src.model\_data\_factory.}}\sphinxbfcode{\sphinxupquote{ModelDataFactory}}}{\emph{\DUrole{n}{request\_path}}}{}
\sphinxAtStartPar
\sphinxstyleemphasis{Model Data Factory}

\sphinxAtStartPar
This class creates an instance of the Model data class with the input of the scheduler engine.
\begin{quote}
\begin{description}
\item[{Attributes:}] \leavevmode
\sphinxAtStartPar
request\_path                 path or json file
data                         model data class

\end{description}
\end{quote}
\index{create() (src.model\_data\_factory.ModelDataFactory static method)@\spxentry{create()}\spxextra{src.model\_data\_factory.ModelDataFactory static method}}

\begin{fulllineitems}
\phantomsection\label{\detokenize{index:src.model_data_factory.ModelDataFactory.create}}\pysiglinewithargsret{\sphinxbfcode{\sphinxupquote{static\DUrole{w}{  }}}\sphinxbfcode{\sphinxupquote{create}}}{\emph{\DUrole{n}{request\_path}}}{}
\sphinxAtStartPar
\sphinxstyleemphasis{Create}
\begin{description}
\item[{This method creates an instance of the model data class with the input json file provided}] \leavevmode\begin{description}
\item[{Attributes:}] \leavevmode
\sphinxAtStartPar
request\_path:                 input json file

\end{description}

\end{description}

\end{fulllineitems}


\end{fulllineitems}



\section{Scheduler Response Factory}
\label{\detokenize{index:module-src.scheduler_response_factory}}\label{\detokenize{index:scheduler-response-factory}}\index{module@\spxentry{module}!src.scheduler\_response\_factory@\spxentry{src.scheduler\_response\_factory}}\index{src.scheduler\_response\_factory@\spxentry{src.scheduler\_response\_factory}!module@\spxentry{module}}\phantomsection\label{\detokenize{index:module-4}}\index{module@\spxentry{module}!src.scheduler\_response\_factory@\spxentry{src.scheduler\_response\_factory}}\index{src.scheduler\_response\_factory@\spxentry{src.scheduler\_response\_factory}!module@\spxentry{module}}\index{SchedulerResponseFactory (class in src.scheduler\_response\_factory)@\spxentry{SchedulerResponseFactory}\spxextra{class in src.scheduler\_response\_factory}}

\begin{fulllineitems}
\phantomsection\label{\detokenize{index:src.scheduler_response_factory.SchedulerResponseFactory}}\pysiglinewithargsret{\sphinxbfcode{\sphinxupquote{class\DUrole{w}{  }}}\sphinxcode{\sphinxupquote{src.scheduler\_response\_factory.}}\sphinxbfcode{\sphinxupquote{SchedulerResponseFactory}}}{\emph{\DUrole{n}{model}}, \emph{\DUrole{n}{data}\DUrole{p}{:}\DUrole{w}{  }\DUrole{n}{{\hyperref[\detokenize{index:src.model_data.ModelData}]{\sphinxcrossref{src.model\_data.ModelData}}}}}}{}
\sphinxAtStartPar
\sphinxstyleemphasis{Scheduler Response Factory}

\sphinxAtStartPar
This class creates an instance of the Schedule Response class with the output of the scheduler engine.
\begin{quote}
\begin{description}
\item[{Attributes:}] \leavevmode
\sphinxAtStartPar
response:                     scheduler data class.

\sphinxAtStartPar
model:                        Scheduler engine class.

\sphinxAtStartPar
data:                         model data class.

\end{description}
\end{quote}
\index{create() (src.scheduler\_response\_factory.SchedulerResponseFactory static method)@\spxentry{create()}\spxextra{src.scheduler\_response\_factory.SchedulerResponseFactory static method}}

\begin{fulllineitems}
\phantomsection\label{\detokenize{index:src.scheduler_response_factory.SchedulerResponseFactory.create}}\pysiglinewithargsret{\sphinxbfcode{\sphinxupquote{static\DUrole{w}{  }}}\sphinxbfcode{\sphinxupquote{create}}}{\emph{\DUrole{n}{model}}, \emph{\DUrole{n}{data}}}{}
\sphinxAtStartPar
\sphinxstyleemphasis{Create}

\sphinxAtStartPar
This method creates an instance of the schedule response data class with the solution provided by the engine
scheduler.
\begin{quote}
\begin{description}
\item[{Attributes:}] \leavevmode
\sphinxAtStartPar
data:                 Receive the model data class of the schedule.

\end{description}
\end{quote}

\end{fulllineitems}


\end{fulllineitems}



\section{Utils}
\label{\detokenize{index:module-src.utils}}\label{\detokenize{index:utils}}\index{module@\spxentry{module}!src.utils@\spxentry{src.utils}}\index{src.utils@\spxentry{src.utils}!module@\spxentry{module}}\phantomsection\label{\detokenize{index:module-5}}\index{module@\spxentry{module}!src.utils@\spxentry{src.utils}}\index{src.utils@\spxentry{src.utils}!module@\spxentry{module}}\index{check\_environment() (in module src.utils)@\spxentry{check\_environment()}\spxextra{in module src.utils}}

\begin{fulllineitems}
\phantomsection\label{\detokenize{index:src.utils.check_environment}}\pysiglinewithargsret{\sphinxcode{\sphinxupquote{src.utils.}}\sphinxbfcode{\sphinxupquote{check\_environment}}}{}{}
\sphinxAtStartPar
\sphinxstyleemphasis{Check Environment}

\sphinxAtStartPar
Check if the python environment is correctly configured

\end{fulllineitems}

\index{initialize\_logger() (in module src.utils)@\spxentry{initialize\_logger()}\spxextra{in module src.utils}}

\begin{fulllineitems}
\phantomsection\label{\detokenize{index:src.utils.initialize_logger}}\pysiglinewithargsret{\sphinxcode{\sphinxupquote{src.utils.}}\sphinxbfcode{\sphinxupquote{initialize\_logger}}}{\emph{\DUrole{n}{name}}}{}
\sphinxAtStartPar
\sphinxstyleemphasis{Initialize Logger}

\sphinxAtStartPar
Initialize the logger functionality to capture the progress of the execution

\end{fulllineitems}

\index{list\_to\_reason() (in module src.utils)@\spxentry{list\_to\_reason()}\spxextra{in module src.utils}}

\begin{fulllineitems}
\phantomsection\label{\detokenize{index:src.utils.list_to_reason}}\pysiglinewithargsret{\sphinxcode{\sphinxupquote{src.utils.}}\sphinxbfcode{\sphinxupquote{list\_to\_reason}}}{\emph{\DUrole{n}{self}}, \emph{\DUrole{n}{exception\_list}}}{}
\sphinxAtStartPar
\sphinxstyleemphasis{List to Reason}

\sphinxAtStartPar
Raise an exception list

\end{fulllineitems}

\index{to\_dict() (in module src.utils)@\spxentry{to\_dict()}\spxextra{in module src.utils}}

\begin{fulllineitems}
\phantomsection\label{\detokenize{index:src.utils.to_dict}}\pysiglinewithargsret{\sphinxcode{\sphinxupquote{src.utils.}}\sphinxbfcode{\sphinxupquote{to\_dict}}}{\emph{\DUrole{n}{df}}, \emph{\DUrole{n}{index}\DUrole{o}{=}\DUrole{default_value}{None}}}{}
\sphinxAtStartPar
\sphinxstyleemphasis{To dict}

\end{fulllineitems}



\chapter{Unit Testing Documentation}
\label{\detokenize{index:unit-testing-documentation}}

\section{Data Test}
\label{\detokenize{index:module-test.test_data}}\label{\detokenize{index:data-test}}\index{module@\spxentry{module}!test.test\_data@\spxentry{test.test\_data}}\index{test.test\_data@\spxentry{test.test\_data}!module@\spxentry{module}}\phantomsection\label{\detokenize{index:module-6}}\index{module@\spxentry{module}!test.test\_data@\spxentry{test.test\_data}}\index{test.test\_data@\spxentry{test.test\_data}!module@\spxentry{module}}\index{DataTest (class in test.test\_data)@\spxentry{DataTest}\spxextra{class in test.test\_data}}

\begin{fulllineitems}
\phantomsection\label{\detokenize{index:test.test_data.DataTest}}\pysiglinewithargsret{\sphinxbfcode{\sphinxupquote{class\DUrole{w}{  }}}\sphinxcode{\sphinxupquote{test.test\_data.}}\sphinxbfcode{\sphinxupquote{DataTest}}}{\emph{\DUrole{n}{methodName}\DUrole{o}{=}\DUrole{default_value}{\textquotesingle{}runTest\textquotesingle{}}}}{}
\sphinxAtStartPar
\sphinxstyleemphasis{Data Test}

\sphinxAtStartPar
This class defines the unitary test for the input data instance of scheduler
\index{test\_connectivity() (test.test\_data.DataTest method)@\spxentry{test\_connectivity()}\spxextra{test.test\_data.DataTest method}}

\begin{fulllineitems}
\phantomsection\label{\detokenize{index:test.test_data.DataTest.test_connectivity}}\pysiglinewithargsret{\sphinxbfcode{\sphinxupquote{test\_connectivity}}}{}{}
\sphinxAtStartPar
\sphinxstyleemphasis{Test Connectivity of the plant configuration}

\sphinxAtStartPar
Test if the equipment of the plant configuration is connected to the rest of the equipment of the workflow

\end{fulllineitems}

\index{test\_consistency() (test.test\_data.DataTest method)@\spxentry{test\_consistency()}\spxextra{test.test\_data.DataTest method}}

\begin{fulllineitems}
\phantomsection\label{\detokenize{index:test.test_data.DataTest.test_consistency}}\pysiglinewithargsret{\sphinxbfcode{\sphinxupquote{test\_consistency}}}{}{}
\sphinxAtStartPar
\sphinxstyleemphasis{Test consistency of the workflows}

\sphinxAtStartPar
Test if the workflows defined are consistent

\end{fulllineitems}

\index{test\_data\_input() (test.test\_data.DataTest method)@\spxentry{test\_data\_input()}\spxextra{test.test\_data.DataTest method}}

\begin{fulllineitems}
\phantomsection\label{\detokenize{index:test.test_data.DataTest.test_data_input}}\pysiglinewithargsret{\sphinxbfcode{\sphinxupquote{test\_data\_input}}}{}{}
\sphinxAtStartPar
\sphinxstyleemphasis{Test Data Input}

\sphinxAtStartPar
Test if some input data instances are feasible plant configuration for the scheduler engine
\begin{description}
\item[{Input data instances:}] \leavevmode\begin{itemize}
\item {} 
\sphinxAtStartPar
example\_test\_1.json

\item {} 
\sphinxAtStartPar
example\_test\_2.json

\item {} 
\sphinxAtStartPar
example\_test\_3.json

\end{itemize}

\end{description}

\end{fulllineitems}

\index{test\_feasible\_time\_horizon() (test.test\_data.DataTest method)@\spxentry{test\_feasible\_time\_horizon()}\spxextra{test.test\_data.DataTest method}}

\begin{fulllineitems}
\phantomsection\label{\detokenize{index:test.test_data.DataTest.test_feasible_time_horizon}}\pysiglinewithargsret{\sphinxbfcode{\sphinxupquote{test\_feasible\_time\_horizon}}}{}{}
\sphinxAtStartPar
\sphinxstyleemphasis{Test feasible time horizon}

\sphinxAtStartPar
Test if the time horizon provided is feasible to allocate the longest schedule

\end{fulllineitems}


\end{fulllineitems}



\section{Engine Test}
\label{\detokenize{index:module-test.test_engine}}\label{\detokenize{index:engine-test}}\index{module@\spxentry{module}!test.test\_engine@\spxentry{test.test\_engine}}\index{test.test\_engine@\spxentry{test.test\_engine}!module@\spxentry{module}}\phantomsection\label{\detokenize{index:module-7}}\index{module@\spxentry{module}!test.test\_engine@\spxentry{test.test\_engine}}\index{test.test\_engine@\spxentry{test.test\_engine}!module@\spxentry{module}}\index{EngineTest (class in test.test\_engine)@\spxentry{EngineTest}\spxextra{class in test.test\_engine}}

\begin{fulllineitems}
\phantomsection\label{\detokenize{index:test.test_engine.EngineTest}}\pysiglinewithargsret{\sphinxbfcode{\sphinxupquote{class\DUrole{w}{  }}}\sphinxcode{\sphinxupquote{test.test\_engine.}}\sphinxbfcode{\sphinxupquote{EngineTest}}}{\emph{\DUrole{n}{methodName}\DUrole{o}{=}\DUrole{default_value}{\textquotesingle{}runTest\textquotesingle{}}}}{}
\sphinxAtStartPar
\sphinxstyleemphasis{Engine Test}

\sphinxAtStartPar
This class defines the unitary test for the engine scheduler
\index{test\_engine() (test.test\_engine.EngineTest method)@\spxentry{test\_engine()}\spxextra{test.test\_engine.EngineTest method}}

\begin{fulllineitems}
\phantomsection\label{\detokenize{index:test.test_engine.EngineTest.test_engine}}\pysiglinewithargsret{\sphinxbfcode{\sphinxupquote{test\_engine}}}{}{}
\sphinxAtStartPar
\sphinxstyleemphasis{Test engine}

\sphinxAtStartPar
Test scheduler engine with different input plant configurations. Check that the output schedule fulfill the
expected functionalities.
\begin{description}
\item[{Input data instances:}] \leavevmode\begin{itemize}
\item {} 
\sphinxAtStartPar
example\_test\_1.json

\item {} 
\sphinxAtStartPar
example\_test\_2.json

\item {} 
\sphinxAtStartPar
example\_test\_3.json

\end{itemize}

\end{description}

\end{fulllineitems}


\end{fulllineitems}



\chapter{Indices and tables}
\label{\detokenize{index:indices-and-tables}}\begin{itemize}
\item {} 
\sphinxAtStartPar
\DUrole{xref,std,std-ref}{genindex}

\item {} 
\sphinxAtStartPar
\DUrole{xref,std,std-ref}{modindex}

\item {} 
\sphinxAtStartPar
\DUrole{xref,std,std-ref}{search}

\end{itemize}


\renewcommand{\indexname}{Python Module Index}
\begin{sphinxtheindex}
\let\bigletter\sphinxstyleindexlettergroup
\bigletter{s}
\item\relax\sphinxstyleindexentry{src.engine}\sphinxstyleindexpageref{index:\detokenize{module-2}}
\item\relax\sphinxstyleindexentry{src.model\_data}\sphinxstyleindexpageref{index:\detokenize{module-0}}
\item\relax\sphinxstyleindexentry{src.model\_data\_factory}\sphinxstyleindexpageref{index:\detokenize{module-3}}
\item\relax\sphinxstyleindexentry{src.scheduler\_response}\sphinxstyleindexpageref{index:\detokenize{module-1}}
\item\relax\sphinxstyleindexentry{src.scheduler\_response\_factory}\sphinxstyleindexpageref{index:\detokenize{module-4}}
\item\relax\sphinxstyleindexentry{src.utils}\sphinxstyleindexpageref{index:\detokenize{module-5}}
\indexspace
\bigletter{t}
\item\relax\sphinxstyleindexentry{test.test\_data}\sphinxstyleindexpageref{index:\detokenize{module-6}}
\item\relax\sphinxstyleindexentry{test.test\_engine}\sphinxstyleindexpageref{index:\detokenize{module-7}}
\end{sphinxtheindex}

\renewcommand{\indexname}{Index}
\printindex
\end{document}